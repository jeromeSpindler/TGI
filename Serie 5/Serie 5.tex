\documentclass{article}

\usepackage{amsmath}
\usepackage{amssymb}
\usepackage{amsfonts}

% \setlength{\parindent}{0pt}

\setcounter{secnumdepth}{0}

\title{Übungsblatt 5 Lösungen}
\author{Jerome Spindler, Niklas Carstensen}

\begin{document}
\maketitle

\section{Hausaufgabe 1}

TODO

\section{Hausaufgabe 2}

a)
 
G = (V, $\Sigma$, P, S) 

V = {S} 

$\Sigma$ = {a,b,(,),*,·} 

P:    S $\rightarrow$ a | b | (S) | S* | S·S | S|S 

S = S 
\\ 
b)

$S \rightarrow S \mid S \rightarrow a \mid S \rightarrow a \mid a$

S $\rightarrow$ S·S $\rightarrow$ a·S $\rightarrow$ a·S* 
$\rightarrow$ a·b*

$S \rightarrow S^* \rightarrow (S)^* \rightarrow (S \mid S)^* 
\rightarrow (a \mid S)^* \rightarrow (a \mid b)^*$

\section{Hausaufgabe 3}

a)

$L_{1} = \{w \in \Sigma^* \mid c \in w \implies 
c = \rightarrow \lor \: c = \downarrow \}$

$L_{2} = \{w \in \Sigma^* \mid \forall i \in 
\mathbb{N} : 2 \leq i \leq \left| w \right|_i : 
(w_i = \uparrow \implies w_{i+1} = \leftarrow \land
w_i = \leftarrow \implies w_{i+1} = \downarrow \land
w_i = \downarrow \implies w_{i+1} = \rightarrow \land
w_i = \rightarrow \implies w_{i+1} = \uparrow \land )\}$

$L_{3} = \{w \in \Sigma^* \mid (\forall u,v \in \Sigma^* :
uv = w : \left| u \right|_\uparrow \geq \left| u \right|_\downarrow) \}$
\\
b)

$L_{1} = (\rightarrow^* \downarrow^*)^* $, was nach 2.42 regulär ist.

$L_{2} = (\downarrow^+\rightarrow^+\uparrow^+\leftarrow^+)^+ 
\mid (\rightarrow^+\uparrow^+\leftarrow^+\downarrow^+)^+
\mid (\uparrow^+\leftarrow^+\downarrow^+\rightarrow^+)^+
\mid (\leftarrow^+\downarrow^+\rightarrow^+\uparrow^+)^+ $, 
was nach 2.42 regulär ist.

$L_{3}$ ist nicht regulär, da man die unbeschränkte 'Haushöhe'
zählen müsste.

\section{Hausaufgabe 4}

TODO

\end{document}
