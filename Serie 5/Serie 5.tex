\documentclass{article}

\usepackage{amsmath}
\usepackage{amssymb}
\usepackage{amsfonts}
\usepackage[ansinew]{inputenc} %Windows Umlaute
\usepackage[latin1]{inputenc}  %Linux Umlaute
\usepackage[applemac]{inputenc}%Mac Umlaute

% \setlength{\parindent}{0pt}

\setcounter{secnumdepth}{0}

\title{Übungsblatt 5 Lösungen}
\author{Jerome Spindler, Niklas Carstensen}

\begin{document}
\maketitle

\section{Hausaufgabe 1}

TODO

\section{Hausaufgabe 2}

a)
 
Definiere KFG G:= (V, $\Sigma$, P, S) mit

V = \{S\} ,

$\Sigma$ = \{a,b,(,),*,·\}$ ,

P:= $\{ S \rightarrow a \mid b \mid (S) \mid S* \mid S·S \mid S|S \}$ ,

S = $\{S\}$ 
\\ 
b)

$S \Rightarrow_{L} S \mid S \Rightarrow_{L} a \mid S \Rightarrow_{L} a \mid a$

S $\Rightarrow_{L}$ S·S $\Rightarrow_{L}$ a·S $\Rightarrow_{L}$ a·S* 
$\Rightarrow_{L}w$ a·b*

$S \Rightarrow_{L} S^*\Rightarrow_{L} (S)^*\Rightarrow_{L} (S \mid S)^* 
\Rightarrow_{L} (a \mid S)^* \Rightarrow_{L} (a \mid b)^*$

\section{Hausaufgabe 3}

a)
Wir nehmen für $L_{1}$ an, dass der erste Buchstabe $\rightarrow$ sein muss.

$L_{1} = \{w \in \Sigma^* \mid w = \epsilon \lor (w_{1} = \rightarrow \land \forall i \in \mathbb{N}:1 \leq  i \leq\ (|w| -1}) \implies$

$((w_{i} = \rightarrow \implies w_{i+1} = \downarrow) \land$

$(w_{i} = \downarrow \implies w_{i+1} = \rightarrow))) \}$ 



Wir nehmen für $L_{2}$ an, dass der erste Buchstabe beliebig ist.

$L_{2} = \{w \in \Sigma^* \mid w_1 \in \Sigma_{\epsilon} \land \forall i \in \mathbb{N} :2 \leq i \leq \left| w \right| \implies$

$((w_{i-1} = \uparrow \implies w_i = \uparrow \lor w_i = \leftarrow) \land$

$(w_{i-1} = \leftarrow \implies w_i = \leftarrow \lor w_i = \downarrow) \land$

$(w_{i-1} = \downarrow \implies w_i = \downarrow \lor w_i = \rightarrow) \land$

$(w_{i-1} = \rightarrow \implies w_i = \rightarrow \lor w_i = \uparrow )\}$

Wir nehmen an, dass der erste Buchstabe $\rightarrow$ sein muss und dass die 

"Dächer" nicht die Breite 0 haben dürfen.

$L_{3} = \{w \in \Sigma^* \mid (\forall u,v \in \Sigma^* :
uv = w : \left| u \right|_\uparrow \geq \left| u \right|_\downarrow) \}$
\\
b)

$L_{1} = (\rightarrow^* \downarrow^*)^* $, was nach 2.42 regulär ist.

$L_{2} = (\downarrow^+\rightarrow^+\uparrow^+\leftarrow^+)^+ 
\mid (\rightarrow^+\uparrow^+\leftarrow^+\downarrow^+)^+
\mid (\uparrow^+\leftarrow^+\downarrow^+\rightarrow^+)^+
\mid (\leftarrow^+\downarrow^+\rightarrow^+\uparrow^+)^+ $, 
was nach 2.42 regulär ist.

$L_{3}$ ist nicht regulär, da man die unbeschränkte 'Haushöhe'
zählen müsste.

\section{Hausaufgabe 4}

a)

G = (V, $\Sigma$, P, S) 

V = {S} 

$\Sigma$ = {0,1} 

P: $S \rightarrow 0S1 \mid 1S0 \mid \epsilon$

S = S 
\\
b)

G = (V, $\Sigma$, P, S) 

V = {S,T,U,V} 

$\Sigma$ = {a,b,c,d} 

P:  

$S \rightarrow aSd \mid T \mid U$

$T \rightarrow aTc \mid V$

$U \rightarrow bUd \mid V$

$V \rightarrow bVc$

S = S 

\end{document}
