\documentclass{article}

\usepackage{amsmath}
\usepackage{amssymb}
\usepackage{amsfonts}
\usepackage[ansinew]{inputenc} %Windows Umlaute
\usepackage[latin1]{inputenc}  %Linux Umlaute
\usepackage[applemac]{inputenc}%Mac Umlaute

% \setlength{\parindent}{0pt}

\setcounter{secnumdepth}{0}

\title{Übungsblatt 5 Lösungen}
\author{Jerome Spindler, Niklas Carstensen}

\begin{document}
\maketitle

\section{Hausaufgabe 1}

Voraussetzung: Sei $L_1 = \{a^{n \cdot m} \mid n,m \in N_{>1} \} \cup \{a, \epsilon\}$ eine Sprache.\\
Behauptung: $L_1$ ist nicht regulär.\\
\\
Beweis: (mittels Pumping-Lemma)\\
Da reguläre Sprachen abgeschlossen gegenbüber Komplementbildung sind,\\
zeige dass $\overline{L_1}$ nicht regulär ist. Es gilt offenbar:\\
$\overline{L_1} = \{a^n \mid \textrm{n ist Primzahl}\}$\\
Sei $p \in \mathbb{N}$ beliebig und $w \in \overline{L_1}$. Wir unterscheiden:\\
(1) $|w| = (p+1)! + 1$\\
(2) $|w| = (p+1)! + (p+ 1) + 1$\\
da wir wissen, dass mindestens eine dieser beiden Zahlen eine Primzahl ist.\\
Demnach existiert zwischen diesen zwei Zahlen eine Lücke mit Größe $\geq p + 1$.\\
Seien $x,y,z \in \Sigma^*$ ,sodass gilt: $w = xyz \land |y| > 0 \land |xy| \leq p$. Es folgt:\\
Zu (1):\\
Behaupte: Es gilt $xy^2z \notin \overline{L_1$.\\
Beweis: Anhand der Definition von $|w|$ wissen wir, dass $|w| -1$ geteilt wird\\
von allen $n \in \mathbb{N}$ für, die gilt $2\leq n \leq p+1$. Somit folgt, dass für alle diese n gilt:\\
$n|(p+1! +n) \Leftrightarrow n|((|w| -1) + n \Leftrightarrow n| (|w| + (n-1))$\\
Berücksichtigen wir nun die -1 für unsere Abschätzung von n folgt:\\
$\forall n \in \mathbb{n}: 1 \leq n \leq p \implies |w| + n \quad \textrm{nicht prim}$\\
Woraus folgt, dass für y mit $1 \leq |y| \leq p$ gilt, dass $xy^2z \notin \overline{L_1}$ .\\
Somit ist $\overline{L_1}$ nicht regulär und auch $L_1$ nicht regulär.\\
Zu(2):\\
Behaupte: Es gilt $xy^0z \notin \overline{L_1}$.\\
Beweis: Anhand der Definition von $|w|$ wissen wir, dass $|w| -1$ geteilt wird\\
von $p+1$. Desweiteren wissen wir, dass für alle $n \in \mathbb{N}$ mit $0 \leq n \leq p-1$ gilt:\\
$((p+1) - n) | ((p+1)! + (p + 1 - n)) \Leftrightarrow ((p+1) -n) | (|w| -1 -n) \Leftrightarrow (p+1) -n | (|w| - (n + 1)$\\
Hieraus folgt unter Berücksichtigung der +1:\\
$\forall n \in \mathbb{N}: 1 \leq n \leq p: |w| - n \quad \textrm{nicht prim}$\\
Woraus folgt, dass für y mit $1 \leq |y| \leq p$ gilt, dass $xy^2z \notin \overline{L_1}$ .\\
Somit ist $\overline{L_1}$ nicht regulär und auch $L_1$ nicht regulär.\\

\section{Hausaufgabe 2}

a)
 
Definiere KFG G:= (V, $\Sigma$, P, S) mit

V = $\{S\}$ ,

$\Sigma$ = \{a,b,(,),*,·\}$ ,

P:= $\{ S \rightarrow a \mid b \mid (S) \mid S^* \mid S·S \mid S|S \}$ ,

S = $\{S\}$ 
\\ 
b)

$S \Rightarrow_{L} S \mid S \Rightarrow_{L} a \mid S \Rightarrow_{L} a \mid a$

S $\Rightarrow_{L}$ S·S $\Rightarrow_{L}$ a·S $\Rightarrow_{L}$ a·S* 
$\Rightarrow_{L}w$ a·b*

$S \Rightarrow_{L} S^*\Rightarrow_{L} (S)^*\Rightarrow_{L} (S \mid S)^* 
\Rightarrow_{L} (a \mid S)^* \Rightarrow_{L} (a \mid b)^*$

\section{Hausaufgabe 3}

a)
Wir nehmen für $L_1$ an, dass der erste Buchstabe $\rightarrow$ sein muss.

$L_{1} = \{w \in \Sigma^* \mid w = \epsilon \lor (w_{1} = \rightarrow \land \forall i \in \mathbb{N}:1 \leq  i \leq\ (|w| -1}) \implies$

$((w_{i} = \rightarrow \implies w_{i+1} = \downarrow) \land$

$(w_{i} = \downarrow \implies w_{i+1} = \rightarrow))) \}$\\

\\

Wir nehmen für $L_2$ an, dass der erste Buchstabe beliebig ist.

$L_{2} = \{w \in \Sigma^* \mid w_1 \in \Sigma_{\epsilon} \land \forall i \in \mathbb{N} :2 \leq i \leq \left| w \right| \implies$

$((w_{i-1} = \uparrow \implies w_i = \uparrow \lor w_i = \leftarrow) \land$

$(w_{i-1} = \leftarrow \implies w_i = \leftarrow \lor w_i = \downarrow) \land$

$(w_{i-1} = \downarrow \implies w_i = \downarrow \lor w_i = \rightarrow) \land$

$(w_{i-1} = \rightarrow \implies w_i = \rightarrow \lor w_i = \uparrow )\}$
\\
\\
\\
\\

Wir nehmen für $L_3$ an, dass der erste und letzte Buchstabe $\rightarrow$ sein 

müssen die "Dächer" nicht die Breite 0 haben dürfen, sofern $w \neq \epsilon$ und

der Abstand zwischen "Häusern" $\geq 1$ ist.

$L_{3} = \{w \in \Sigma^* \mid (\forall u,v \in \Sigma^* :
uv = w : \left| u \right|_\uparrow \geq \left| u \right|_\downarrow) \land $

$(w = \epsilon \lor (w_1 = \rightarrow \land w_{|w|} = \rightarrow \land $

$\forall i_1 , i_2 \in \mathbb{N}: (1 < i_1 < i_2 < |w| \land (w_{i_1} = \uparrow
\land w_{i_2} = \downarrow )\lor (w_{i_1} = \downarrow
\land w_{i_2} = \uparrow )  ) \implies$

$\exists j \in \mathbb{N}: i_1 < j < i_2 \land w_j = \rightarrow ) \}$
\\
b)

$L_{1} = (\rightarrow (\rightarrow \downarrow)^*)^* $, was nach 2.42 regulär ist.

$L_{2} = (\downarrow^+\rightarrow^+\uparrow^+\leftarrow^+)^* 
\mid (\rightarrow^+\uparrow^+\leftarrow^+\downarrow^+)^*
\mid $

$(\uparrow^+\leftarrow^+\downarrow^+\rightarrow^+)^*
\mid (\leftarrow^+\downarrow^+\rightarrow^+\uparrow^+)^* $, 
was nach 2.42 regulär ist.

$L_{3}$ ist nicht regulär, da man die unbeschränkte 'Haushöhe'zählen und für

jeden $\uparrow$ einen $\downarrow$ Partner finden müsste.
\\
c)

$G_1 = (V_1 , \Sigma_1 , P_1 , S_1) $ mit

$V_1 = \{S,T\}$

$\Sigma_1 = \{ \rightarrow , \downarrow , \epsilon \}$

$P_1 = \{$$

$\quad \quad S \rightarrow \quad \rightarrow T \mid \epsilon ,$

$\quad \quad T \rightarrow \quad \downarrow \rightarrow T \mid \epsilon \}$

$S_1 = {S}$
\\

$G_2 = (V_2 , \Sigma_2 , P_2 , S_2) $ mit

$V_2 = \{U,L,D,R\}$

$\Sigma_2 = \{\rightarrow , \leftarrow , \uparrow , \downarrow , \epsilon \}$

$P_2 = \{$

$\quad \quad U \rightarrow \quad \uparrow U \mid \uparrow L  \mid \epsilon ,$

$\quad \quad L \rightarrow \quad \leftarrow L \mid \leftarrow D \mid \epsilon ,$

$\quad \quad D \rightarrow \quad \downarrow D \mid \downarrow R \mid \epsilon ,$

$\quad \quad R \rightarrow \quad \rightarrow R \mid \rightarrow U \mid \epsilon \}$


$S_2 = \{U, L, D, R\}$
\\

$G_3 = (V_3 , \Sigma_3 , P_3 , S_3) $ mit

$V_3 = \{S,T,R\}$

$\Sigma_3 = \{\rightarrow , \uparrow , \downarrow \}$

$P_3 = \{$

$\quad \quad S \rightarrow \quad R T S \mid R ,$

$\quad \quad T \rightarrow \quad \uparrow T \downarrow \mid \uparrow R \downarrow }$

$\quad \quad R \rightarrow \quad \rightarrow \mid \rightarrow R\}$

$S_3 = \{S\}$
\\
\\

\section{Hausaufgabe 4}

a)

G = (V, $\Sigma$, P, S) mit

$V = \{S\}$ 

$\Sigma = \{0,1, \epsilon \} $

P: $S \rightarrow 0S1 \mid 1S0 \mid \epsilon$

S = S 
\\
b)

G = (V, $\Sigma$, P, S) mit

$V = \{S,T,U,V\}$ 

$\Sigma = \{a,b,c,d, \epsilon\}$ 

$P: \{$  

$\quad \quad S \rightarrow \quad aSd \mid T \mid U ,$

$\quad \quad T \rightarrow \quad aTc \mid V ,$

$\quad \quad U \rightarrow \quad bUd \mid V ,$

$\quad \quad V \rightarrow \quad bVc \mid \epsilon \}$

$S = \{S\}$ 

\end{document}
